\section{Site cost estimation $\leq 1$ page.}
%Explain the importance of a common method to estimate the costs for sites
%given the resource needs of the experiments.
%Describe Renaud's model and future perspectives for this area.

\subsection{Context}
%The IT resources deployed in the data centres contributing to WLCG are an important source of expense for funding agencies.
%With the expected amount of data to be recorded at HL-LHC, it is highly desirable for WLCG to estimate the
%amount of IT resources that will be available for data processing in sites over time.
%Simple projections from the current trends in the IT market, assuming a constant budget in sites over time, indicate that
%the available amount of resources across the WLCG will not be sufficient at all to process the HL-LHC data (reference?).
%Therefore, it becomes clear that, unless technology revolutions happen in the meantime,
%LHC experiments will need to make a different usage of IT resources that sites will deploy.

The purpose of site cost estimation is to understand and measure, at a global scale,
what the data centres typical expenses are, and predict what they may be in years from now.
The approach chosen in this context is to model those expenses, taking into account the diversity of national contexts
across sites in terms of funding, procurement procedures, and local market conditions.
These results will help experiments plan new computing strategies that
will improve the cost-effectiveness of their resource usage.

\subsection{First results}

Some of the first elements to address are the diversity of expenses across sites, and the definition of what these
expenses are.
A simple and quick exercice was made by four candidate sites, aiming to get a first estimate of the financial cost to run
a given workflow and to store a given amount of data, in terms of IT resources and power consumption.
The answers to this exercice happened to be significantly different from site to site, up to a factor of 2.
The reasons were found to originate from different aspects: the intrinsic variety of costs, the measurement method, and
the understanding of what a given metric means.

Those results showed clearly the need of a consolidated and common model and method to measure costs across
WLCG data centres.
Here are a few examples to illustrate this point. One may consider the cost of a server providing a given capacity,
including (or not) the equipement that is shipped with it (rack, switch, adaptor etc.).
One may also consider that one capacity unit of tape storage includes (or not) the investments made in
library, drives, disk cache etc.
that are needed for such system to work properly.
Finally, the measurement of an electrical consumption may include (or not) the Power Usage Effectiveness
of the data center
in order to take into account UPS or HVAC system contributions to the final power bill.

Consequently, it becomes fundamental in the context of site cost
estimation to establish precise definition of the cost-related metrics in order to build a reliable model.

\subsection{Next steps}

A preliminary attempt to address data center resource TCO through cost modeling was shown in reference~\cite{costmodel}.
This model assumes that a data centre invests year after year a constant budget in its following assets:
batch system, disk storage and tape system capacities.
If this hypothesis is satisfied (even roughly), one can show that budget ($B$) and available capacity ($K$) over time are bound
together by a quantity ($c^*$) that depends on hardware cost evolution and lifetime. This model however does not address all the components of a TCO, like manpower.

\begin{equation}
    B (t) = K (t) \times c^* (t)
    \label{eq:costmodel}
\end{equation}

Site cost studies may leverage this model to estimate the financial impact of a variation in the usage that experiments
make of data centre IT resources.
But the quantity $c^*$ may differ significantly from site to site, so an extension of this model at global scale
will have to take into account as many as site-dependent parameters as possible to establish $c^*$.
