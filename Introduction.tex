\section{Introduction}
The computing infrastructure for the LHC experiments, managed via the
WLCG project~\cite{wlcg}, has been successfully operating since 2008,
with a good match between the resources needed by the experiments and
those made available by the funding bodies. In the last few years
though it became increasingly clear that Run 3 (for ALICE and LHCb)
and Run 4, or HL-LHC (for ATLAS and CMS) will determine very
significant increases in the scale of computing, which will not be
simply accommodated by technological improvements in the likely
scenario of a flat budget evolution. Simply extrapolating the current
software performance to the expected trigger rates and average pile-up
produces a $O(10)$ discrepancy between the needed and the affordable
levels of computing and storage capacity. For this reason,
``revolutionary'' changes in the software and the computing models of
the experiments will be absolutely necessary.

The need of a joint working group between WLCG and the
HSF~\cite{hsf} dedicated to the study of performance and cost of
computing became apparent and it finally started at the end of
2017, with a long term roadmap that extends to the start of
HL-LHC. About thirty members, from experiments, sites and IT and software
experts participate to the group activities. The initial focus has
been on improving the understanding of current workloads and to
establish methodologies and tools to analyse their performance; however,
some thought has already been given to the exploration of future
scenarios and to try to quantify the gains that could be achieved
through paradigm changes. Currently the most important areas of work are:
\begin{itemize}
\item the collection of reference workloads from each experiment, to conduct
  performance studies in controlled and repeateable conditions;
\item the definition of the metrics that best characterise the applications and the implementation of tools to measure them;
\item the adoption of a common framework for estimating resource needs:
\item the adoption of a common process to evaluate the cost of an infrastructure
  as a function of the experiment needs;
\item preliminary studies to explore possible savings in different areas.
\end{itemize}

